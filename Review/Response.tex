\documentclass[11pt]{bgcletter}
\usepackage{hyperref}
\usepackage{textcomp}
\usepackage{amsmath}

\name{Dr.\ Carlos A. Sierra}
\signature{ \vspace{-2cm} Carlos A. Sierra, PhD}
\email{csierra}
\telephone{6133}
\begin{document}
\begin{letter}{Dr. A. J. T. Jull \\
 Editor in Chief \\ Radiocarbon}

\opening{Dear Dr. Jull,}
Thank you very much for your consideration of my manuscript and for willing to accept it subject to minor revisions. I made several changes to the manuscript based on the reviewers' comments, which I describe in detail in this letter. 

In the text below, I provide answers ({\color{blue} blue font}) to all reviewers' comments ({\it italics font}). 

{\bf Reviewer 1} \\
{\it The ms presents an investigation of the rate of decline in atmospheric 14C (including trend and amplitude of seasonal cycles) for the last 40 years in both hemispheres and forecasting the timing when atmospheric 14C crosses the pre-bomb level. This is an interesting research, whose results are very useful not only for the study of the carbon cycle and influences of fossil fuel emissions but also for radiocarbon dating of recent organic samples. The analysis presented and the topic discussed in this ms are suitable for Radiocarbon. The paper is well-written with adequate abstract. However, there are several minor issues mainly in the presentation, which need to be improved. I therefore recommend the paper should be acceptable for publication after addressing the below minor points.
}

{\color{blue} Thanks for recognizing the value of the manuscript and pointing out issues that need to be addressed before publication. }

\begin{enumerate}
\item {\it Line 124, the author uses the formula $\Delta$14C = (F-1) * 1000 to convert $\Delta$14C to F. Strictly speaking this formula is not relevant here. All the $\Delta$14C values reported by Hua et al. (2013) for the post-bomb period and Reimer et al. (2013, IntCal13) for the pre-bomb period are age-corrected or decay-corrected $\Delta$14C. 

The correct formula therefore should be $\Delta$14C = (F*exp($\lambda$*(1950-x))-1) * 1000 , with $\lambda$ = 1/8267 yr-1 and x being the 14C decay constant and the time (month, year) of growth (eg, tree rings) or time of sampling (eg., atmospheric samples), respectively. Although F values derived from the two equations are not much different, a correct formula must be used.}

{\color{blue} This is an important point, and I am grateful that both reviewers pointed it out. 

In the previous version of the manuscript, I performed all calculations using fraction modern $F$ as described, but now I realized that this would undo the correction to the OX1 standard introduced by the $\Delta^{14}$C notation. For this reason, I changed my calculations to use the \emph{absolute fraction modern} $F'$, which takes $\Delta^{14}$C and expresses it as a fraction through the formula $F' = (\Delta^{14} \text{C} / 1000) +1$. In this way, the fractions are expressed with respect to the value of the standard in 1950 that does not change over time. 

This change had a minor effect on the results. The ets models fitted to the series had no major changes and the forestcasts were modified only by a few years. 
}

\item { \it Equation (1), the author should clearly mention what ``s'' is.
Similarly, what is the meaning of the subscript ``m'' of ``st-m'' in section 3 (p.8)?
Are $\alpha$, $\beta$ and $\gamma$ in p.8 constant parameters? If so, please clearly mention.}

{\color{blue} This information was added to each equation.}

\item {\it Lines 175-178, the author gives an example on the rate of decline for the last year of Hua et al. (2013) data of -2.4 \textperthousand \ and -2.6 \textperthousand \ for the northern and southern hemisphere, respectively.
Why are the rates of the only last year of the compiled data mentioned here? Instead, such rates of the last 3 or 5 years should be discussed here to show a clear trend of atmospheric 14C. One-year data are too short for this purpose. }

{\color{blue} Originally, I wanted to report the measured slope for the last 12 months of the series because the last year in each of them was not complete. To address this comment, I calculated the slope for each of the series from 2005 until the end of each series. This is now reported in a separate table. }

\item {\it The author presents new measurements of recent plants growing in different locations in Figure 6. As these data are of great interest, the author should report these values in tabulated form in this paper as well. }

{\color{blue} This table was provided in the supplementary material as a comma delimited file, but now I created a new table in the manuscript that presents the same information. }

\item {\it Line 38, ``atmospheric radiocarbon concentrations''. Strictly speaking, 14C values in the form of F14C or $\Delta$14C presented in this ms are not 14C concentration. ``atmospheric radiocarbon content'' should be used instead. The same comment applies for all ``atmospheric radiocarbon concentrations'' mentioned in the paper.}

{\color{blue} Changed as suggested.}

\item {\it Lines 59-60, there is a typo in ``... d) it sets a new reference point for ...''. ``d)''  should be replaced by ``c)''.}

{\color{blue} Fixed.}

\item {\it Lines 75-77, ``Standard atmospheric radiocarbon curves are only released to the scientific community at irregular intervals (Hua and Barbetti, 2004; Hua et al., 2013), ...''. ``Compiled atmospheric radiocarbon data'' or ``Observed atmospheric radiocarbon curves'' is much better than ``Standard atmospheric radiocarbon curves''. The same comment applies for all ``Standard atmospheric radiocarbon curves'' stated in the paper.}

{\color{blue} Changed as suggested.}

\item {\it There are no ``Figure 5'' quoted in the text.}

{\color{blue} Fixed.}

\item  Figure 1 (1a-b), there are no units for $\Delta$14C in the 2nd y-axes.

{\color{blue} Fixed.}

\end{enumerate}

{\bf Reviewer 2} \\
{\it This manuscript reports a forecast of the 14CO2 declining trend in both northern and southern hemispheres based solely on mathematical simulations on the compiled 14CO2 time series from 1975 to 2010 from Hua et al (2013).  The author first obtained the best fit model output (with a set of parameters) from a set of 30 different time series decomposition models that simulate the observations, then he applied the exponential smoothing of the best model to obtain the forecast.  He states $\Delta$14C is decreasing at a rate of -2.4 and 2.6 \textperthousand \ per year last year for the S and N hemispheres, and predicts that $\Delta$14C will pass zero, the pre-1950 level around 2030 in the N hemisphere, and around 2040 in the southern hemisphere. Because this forecast is based on compiled data sets so it gives a spatially integrated average trend for the two hemispheres.  However, a specific site could be different from it, depending on its location. The disadvantage of the forecast is that it is only based on mathematical simulations from the past observations and it does not include any processes or emission predictions.  Since the processes controlling the changes of the $\Delta$14C over time are different, this prediction can easily be off.  }

{\color{blue} Thanks for the comments and pointing out a potential disadvantage of the method I introduced. A process-based forecast is obviously desirable, and Graven (2015, PNAS 112:9542) has done this already. However, a forecast based on a process-based model also has uncertainties due to the model structure and parameterization, fossil-fuel and land-use emission scenarios, and future anthropogenically induced nuclear activity, among others. Both process-based and observations-based forecasts have disadvantages, but they complement each other. Since an observation-based forecast such as the one I present here has not been published before, I believe it adds to our understanding of current and future radiocarbon dynamics. }

{\it The forecast of the time when the $\Delta$14C will pass zero is useful for the application of 14C since after this turning point we can no longer determine if a sample is from before or after 1950 with only one single measurement. This mathematical prediction may serve as a reference or a checkpoint for various global carbon cycle models. The forecast is also useful before the official bomb curves (calibration curves) are available. The subject is certainly of interests to the general radiocarbon community and thus is suitable for publication in the journal of Radiocarbon. The manuscript was written clearly and is consistent in general. However I do think there are a couple points in the paper that were not well presented and should be clarified and improved. I therefore recommend the paper should be accepted for publication after minor revision.}

{\color{blue} Thanks for the recommendation. In the points below I provide answers to your comments, which has resulted in an improved manuscript. }

{\it 1) The uncertainty ranges of the predictions seem quite high, especially with time. Although the author included the uncertainty ranges of the forecast in the figures, it would be more convenient for the readers if they were also indicated in the text, as $\pm$ 1 sigma as well. The predicted $\Delta$14C declining rates of -2.4 \textperthousand \ yr and -2.6\textperthousand \ yr for last year are much lower than the observations, though the author claim that they are ``within forecast uncertainty range''.  This doesn't necessarily mean the prediction is good because the ``seemly agreement'' is due to the larger uncertainty ranges. I think the author should discuss about the usefulness of the prediction and the uncertainty ranges, especially when they are much larger than our 14C analytical errors. }

{\color{blue} The uncertainty ranges are relatively high because of a) the nature of the method and b) the reporting choice. First, the forecasting method recursively applies the ETS model to the last observation that already has an uncertainty value, therefore the uncertainty of successive forecasts always increases. I believe this is an advantageous property of the method rather than a disadvantage. It accounts for the fact that it is increasingly more difficult to successively predict the future; and therefore, our confidence in successive predictions should decrease. Second, I chose to report 95 and 80\% prediction intervals. Although the 95\% limit is common practice, the 80\% limit was rather arbitrary. Following the reviewer suggestion, I now report 68\% prediction intervals, which corresponds to one standard deviation of the predictions. Also, I included a discussion on the uncertainty of the method, and added to the paragraphs that discusses prediction uncertainty.}

{\it 2) In Line 215, the author states that ``The observations from these stations follow relatively well the forecasted seasonal cycle, however they are below the forecasted mean. One likely explanation for this difference is the potential contribution of fossil-fuel derived carbon to these central European stations''.  I believe Jungfraujoch site is considered as a clean background site, and its distribution over time has been consistent with other background 14C sites in the N hemisphere in the last decade.  The 14C record in Jungfraujoch site is an important part of the compiled bomb curve that the author used to generate the forecast.  Thus I am confused why the observation at Jungfraujoch site is below the forecasted mean (?) Is the 14C time series at Jungfraujoch site (1975 -- 2010) lower than the compiled observation used for the forecast? It doesn't look like so from Figure 1a.}

{\color{blue} The reviewer is right here. The data from the Jungfraujoch site agrees well with both the mean and the seasonality of the northern hemisphere forecast. It is only the Schauinsland series that agrees only in the seasonality, but not with the mean. I made changes to the text to clarify this point. }

{\it Specific comments

Line 113: Change ``proposed to decomposed'' to ``proposed to decompose''}

{\color{blue} Done.}

{\it Line 124: ``using the relation $\Delta$14C = (F -1)  1000 to convert among different reporting conventions (Reimer et al., 2004)''. This is not correct, according Stuiver and Polach?s definition, $\Delta$14C is known age corrected, which was stated in the paper by Reimer et al., 2004 as well:

$\Delta$14C = (F · exp(1950-y)/8267 -1) · 1000

Where y = year of sample growth or collection = year of measurement.
Thus the conversion from $\Delta$14C to F should consider the year of sample.}

{\color{blue} Thanks for pointing this out. As mentioned in the answer to Reviewer 1 on this topic, I now use \emph{absolute fraction modern} to keep the correction applied in the $\Delta^{14}$C term, but reporting in fractional units. }

{\it Line 126: should yt be $\mu$t?}

{\color{blue} No, the observations are designated as $y_t$, which are the result of the mean $\mu_t$ and the error term $\varepsilon_t$.}

{\it In Equation (1), what is s? Same as S, the seasonal term? It was not clearly specified in the text.}

{\color{blue} It is the seasonality term. I added a better description of these different terms in the text.}

{\it Line 140: ``I also used radiocarbon analyses in organic material from annual plants?'', take out ``organic materials from'' since you used the bulk plants. }

{\color{blue} Done.}

{\it For the best model in the time series decomposition, please indicate what the subscripts are: t, t-1, t-m}

{\color{blue} Done.}

{\it Line 173: Figure 2 should be Figure 2b}

{\color{blue} Done.}

{\it Line175: Add ``Figure 2a'' after ``for the north''}

{\color{blue} Done.}

{\it Line 177: ``as the sum of the last 12 bt terms of'', please clarify how long is 1 bt term? 1 month, so 12 terms is one year? Also it is better to include the uncertainties in the declining rates, such as -2.4$\pm$x.x \textperthousand /yr.}

{\color{blue} To address previous comments, I changed this paragraph considerably. I do not report anymore the sum of the last twelve months of the series, but rather the annual slopes for both series since 2005. This information is now included in Table 1, $\pm$ the residual term $\varepsilon_t$ of the model.}

{\it Line 215: The observations from these stations follow relatively well the forecasted seasonal cycle, however they are below the forecasted mean. One likely explanation for this difference is the potential contribution of fossil-fuel derived carbon to these central European stations ? this may not be a good explanation for Jungfraujoch site. Is the 14C time series at Jungfraujoch site lower than the compiled observation used for the forecast?}

{\color{blue} Yes, this is not case for Jungfraujoch, and as mentioned above, I modified this paragraph to clarify this point. }

{\it Line 315:  Change ``fossil-fuel radiocarbon'' to fossil-fuel derived carbon.}

{\color{blue} Done.}

{\it Figure 5 is not in the text?}

{\color{blue} A reference to Figure 5 is now included in the text.}

\vspace{2em}
I hope this new version adequately addresses reviewer's comments and it is now suitable for publication.

\closing{Sincerely, \\
 \includegraphics[scale=0.7]{../../../../Documents/Personal/firma.jpg}
 }
 \end{letter}

 \end{document}
